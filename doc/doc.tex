% Save as dfa.tex and compile with: pdflatex dfa.tex
\documentclass[12pt]{article}
\usepackage[linesnumbered,ruled,vlined]{algorithm2e}
\usepackage{graphicx}
\usepackage{hyperref}
\usepackage[margin=2.5cm]{geometry}
\usepackage{amsmath}

\DeclareMathOperator*{\lse}{lse}
\DeclareMathOperator*{\softmax}{softmax}

\begin{document}

\title{SFC Course Project Documentation \\ Localization using a Modern Hopfield Network}
\author{Jakub Bláha \\ \texttt{xblaha36@stud.fit.vutbr.cz}}
\date{}
\maketitle

\section{Introduction}

This project demonstrates robot localization on a 2D map using a Modern
Hopfield Network for pattern matching. The core problem is to determine a
robot's position based on its local visual observations of the environment.

The approach works in two phases: a \textbf{setup phase} where the robot
explores the map and builds a memory database of position-observation pairs,
and a \textbf{query phase} where the robot uses a new observation to retrieve
its approximate position from memory. The robot captures small visual patches
(e.g., strips of pixels representing what a simple camera would see) at various
known positions. These observations, along with their corresponding positions,
are stored as patterns in the network's memory.

When presented with a new observation, the network performs energy minimization
to retrieve the most similar stored patterns. The robot's position is then
estimated based on the top-k most similar observations, effectively localizing
the robot by combining information from the best matching visual features. The
implementation provides an interactive visualization showing the map, the
robot's movement, its field of view, and the real-time localization process.
Users can configure network parameters such as temperature ($\beta$), number of
observations stored, observation dimensions, and the number of retrieval
iterations.

\section{Modern Hopfield Networks}

Hopfield Networks are associative memory models that can store and retrieve
patterns. Classical Hopfield networks had limited storage capacity
(approximately 0.14N patterns for N neurons) and often retrieved incorrect
patterns called spurious attractors.

Modern Hopfield Networks\footnote{Ramsauer, H., Schäfl, B., Lehner, J., et al.
(2020). Hopfield Networks is All You Need. arXiv:2008.02217.
\url{https://arxiv.org/abs/2008.02217}}\footnote{Modern Hopfield Network.
Wikipedia. \url{https://en.wikipedia.org/wiki/Modern_Hopfield_network}}
significantly improve upon the classical version by using a different energy
function that allows storing exponentially more patterns while providing better
retrieval accuracy. The network works by storing patterns in memory during a
learning phase, then retrieving the most similar patterns when presented with a
new query.

A key parameter is the temperature ($\beta$), which controls how selective the
retrieval is. Higher values make the network focus on fewer best matches, while
lower values allow it to consider a broader range of similar patterns. This is
useful when the robot's observation might match several nearby positions on the
map.

\section{Implementation}
% Architecture overview
% Data structures (map representation, observations, memory)
% Setup phase: building the observation database
% Query phase: position retrieval given new observation
% Algorithm pseudocode

\section{User Manual}
\subsection{Requirements}
% Python version, required libraries, system requirements

\subsection{Compilation and Installation}
% How to install dependencies
% Any setup steps needed

\subsection{Running the Program}
% Command-line interface
% Configuration file format
% Interactive mode usage

\subsection{Program Control}
% Available commands during execution
% How to perform step-by-step execution
% How to interrupt/terminate
% Output interpretation

\section{Results and Discussion}
% Example runs with screenshots/figures
% Accuracy analysis
% Performance characteristics
% Limitations and possible improvements

\section{Conclusion}
% Summary of achievements
% Lessons learned

% \bibliographystyle{plain}
% \bibliography{references}

\end{document}
